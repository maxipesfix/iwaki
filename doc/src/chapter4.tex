\chapter{Recipes}
\label{Chapter4}

In Section~\ref{sec:Ch3Recipes}, we outlined the structure of a recipe. In this chapter we will describe each of the recipe components in detail. You will know how to write a recipe to respond to a particular input, and how to implement control flow, including conditionals via backchaining.



\section{Preconditions}
\section{Body}

A body of the recipe consists of a sequence of \textsl{body elements}. A body element is an \textsl{assignment}, an \textsl{action}, or a \textsl{goal}. At present (version 0.1), body elements are executed sequentially. Action and goal body elements are different from the assignment in that they may take undetermined amount of time. To allow for the execution to continue even if an action or a goal takes too long time to complete, action tag has an optional timeout attribute. Actions are different from both assignments and goals in that their execution may return a status different from the successful \textsl{completed} status. Returned action status can be used to specify the control flow for the rest of the recipe, such as moving to the next body element, moving to the assignpost, or moving to the end (purging the recipe).

\subsection{Assignments}
\subsection{Actions}
\subsection{Goals}

A goal specifies the set of recipes that can be used to backchain on it. There are two way to define a goal. First, via a formula that must be satisfied by the child recipe. Second, by explicitely listing the names of potential child recipes. In the latter case, the formula is optional. We refer to a goal without a formula as an \textsl{empty goal}.

\subsubsection{A goal with a formula}

An example of a goal with a formula is show below:

\lstset{language=XML}
\begin{lstlisting}
    <goal recipe_name="any">
      <atom>
        <slot name="q2u_how_are_you_handled" val="true"/>
        <slot name="this" var="present_user_atom"/>
      </atom>
    </goal>
\end{lstlisting}


During the preprocessing stage, for each of the goals, Iwaki idetifies recipes whose postconditions, once executed, will satisfy the goal's formula.  These recipes are selected from those listed in the \textsl{recipe\_name} attribute. If the \textsl{recipe\_name}  attribute is empty, or is equal to 'any', then all recipes loaded via init file are potential candidates. 
 
\subsubsection{An empty goal}

\lstset{language=XML}
\begin{lstlisting}
    <goal recipe_name="q2u_how_are_you_answer_yes, q2u_how_are_you_answer_no"/>
\end{lstlisting}

An empty goal does not include a formula. Potential child recipes are specified by \textsl{recipe\_name} attribute. If \textsl{recipe\_name} is left empty or set to ``any'' a warning will be issued at the preprocessing stage, since that means that any recipe can be a potential child.

\subsubsection{Control flow with the goals}

When execution reaches the goal elemenet of a recipe's body, the further control flow depends on whether the goal contains a formula or not, and whether the \textsl{forced} attribute of goal tag is set to ``true'' or not. 
\begin{itemize}
\item If the goal's formula is not empty \textit{and} the goal's \textsl{forced} attribute is not set to ``true'', the formula will be checked for satisfiability against current global state of the system. If the formula is already satisfied, the execution flow moves to the next body element (or the postcondition if the goal was the last element of the body). If the formula is not satisfied, backchainable recipes will be attempted, in order of their priority, to be backchained upon by matching them against the goal and checking if their preconditins are satisfied.
\item If the goal's formula is not empty \textit{and} the goal's \textsl{forced} attribute is set to ``true,'' then the goal's formula is \textit{not} checked on satisfiability by current state, and the attempt to backchain is forced independently on whether the goal's formula is already satisfied.
\item If the goal's formula is empty, then the attempt to backchain is made on the backchainable recipes specified in the goal's \textsl{recipe\_name} attribute. This can be thought as a forced backchaining on a formula that is already satisfied, since an empty formula is true by the definition.
\end{itemize}

Why would we want the forced backchaining on a non-empty goal? Consider the following example. We would like to backchain on a set of recipes that are conditioned an uninstantiated user atom and execute some action for that particular user, once it gets instatiated. The parent recipe (the one that contains the goal) may want to constrain the children recipes to be instantiated (and their preconditions to be checked against) a particular user atom. In that case, the goal may look as follows: 

\lstset{language=XML}
\begin{lstlisting}
    <goal recipe_name="q2u_how_are_you_answer_yes, q2u_how_are_you_answer_no" forced="true">
      <atom>
        <slot name="this" var="present_user_atom"/>
      </atom>
    </goal>
\end{lstlisting}

The postconditions of \textsl{q2u\_how\_are\_you\_answer\_yes}, \textsl{q2u\_how\_are\_you\_answer\_no} recipes may be something like this:

\lstset{language=XML}
\begin{lstlisting}
  <assignpost>
    <atom>
      <slot name="this" var="present_user_atom_of_child_recipe"/>
    </atom>
  </assignpost>
\end{lstlisting}

Since the atom in the goal's formula in this example will match any atom of postcondition (because \textsl{this} slot of postconiditon is always uninstantiated before the actual backchained recipe is loaded), more slots may be necessary if there is an ambiguity between the goal's and postcondition's atoms.

\section{Postconditions}

